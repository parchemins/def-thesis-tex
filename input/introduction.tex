\section{Introduction}

\begin{frame}{Introduction}{Éditeur collaboratif temps réel}

  \begin{textblock*}{\textwidth}(-1cm,-3cm) 
    \includegraphics[width=1.19\textwidth]{img/googledocs.png}
  \end{textblock*}
  
\end{frame}


\begin{frame}{Introduction}{Problèmes de confidentialité}
  
  Problèmes de \textbf{confidentialité}, \textbf{censure},
  \textbf{intelligence économique}, \textbf{legislation}, etc.

  \vspace{0.5cm}
  \begin{center}
    \includegraphics[width=0.5\textwidth]{img/centralizedethicproblems.png}
  \end{center}
  
  \vspace{0.5cm}

  \textit{En 2013, les révélations sur PRISM montre que la NSA possède des
    accès aux données hébergées par Google, Facebook, YouTube, Microsoft,
    Yahoo!, Skype, AOL et Apple.}

\end{frame}

\begin{frame}{Introduction}{Problèmes de passage à l'échelle}
  
  Problèmes de passage à l'échelle, notamment en \textbf{nombre de
    collaborateurs}.
  
  \vspace{0.5cm}
  
  \begin{center}
    \includegraphics[width=0.5\textwidth]{img/centralizedcpuproblems.png}
  \end{center}
  
  \vspace{0.25cm}

  \textit{En 2013, Coursera rassembla 41000 étudiants sur un seul cours.  Les
    limitations de l'outil collaboratif utilisé conduisirent au \og
    désastre\fg\footfullcite{strauss2013how}.}

  \vspace{0.25cm}

\end{frame}


\begin{frame}{Introduction}{Ce que l'on veut : un éditeur collaboratif \ldots}
  
  \begin{minipage}{0.45\textwidth}
    \hfill \YES{\cmark} \textbf{Temps réel}
  \end{minipage}
  \begin{minipage}{0.45\textwidth}
    \includegraphics[width=0.75\textwidth]{img/watch.jpg}
  \end{minipage}
    
  \vspace{-0.75cm}

  \begin{minipage}{0.45\textwidth}
    \hfill  \includegraphics[width=0.8\textwidth]{img/toile.jpg}
  \end{minipage}  
  \begin{minipage}{0.45\textwidth}
    \textbf{Web} \YES{\cmark}
  \end{minipage}
  
  \vspace{-0.75cm}
  
  \begin{minipage}{0.45\textwidth}
    \hfill \NO{\xmark}\textbf{Sans fournisseur de services}
  \end{minipage}
  \begin{minipage}{0.45\textwidth}
    \includegraphics[width=0.8\textwidth]{img/service.jpg}
  \end{minipage}

  \vspace{-0.75cm}

  \begin{minipage}{0.45\textwidth}
    \hfill \includegraphics[width=0.8\textwidth]{img/crowd.jpg}
  \end{minipage}
  \begin{minipage}{0.45\textwidth}
    \textbf{Des milliers d'utilisateurs éditant simultanément} \NO{\xmark}
  \end{minipage}


  \only<2>{
    \begin{tikzpicture}[remember picture,overlay]
      \draw (150pt,120pt) node
      [draw,align=left,fill=termithgreen!30,font=\large]{\textbf{L'édition
          collaborative temps réel}\\\textbf{est-elle possible sur le
          Web},\\\textbf{sans l'intervention d'un tiers}\\\textbf{et sans limites quant
        aux dimensions du système ?}};
    \end{tikzpicture}
  }
\end{frame}


\begin{frame}{Introduction}{Un éditeur collaboratif dans les navigateurs}
  

  \begin{minipage}{0.69\textwidth}
    \CRATE est un éditeur collaboratif 
    \begin{itemize}
    \item temps réel \YES{\cmark}
    \item fonctionnant dans les navigateurs Web \YES{\cmark}
    \item sans fournisseur de services \YES{\cmark}
    \item passant à l'échelle \YES{\cmark}
    \end{itemize}
  \end{minipage}
  \begin{minipage}{0.3\textwidth}
    \includegraphics[width=\textwidth,interpolate=false]{img/crateicon.png}
  \end{minipage}
    
  
  \begin{textblock*}{1.18\textwidth}(-1cm,1cm)
    \includegraphics[width=1\textwidth]{img/tmp-52.png}
  \end{textblock*}
  
  \vspace{1cm}

\end{frame}


\begin{frame}{Introduction}{Fonctionnement décentralisé}
  
  \hspace{-1cm}
  \begin{minipage}{0.54\textwidth}
    \begin{itemize}
      \item Chaque éditeur possède une copie locale du document;
      \item Chaque éditeur est connecté à d'autres éditeurs.
      \vspace{0.5cm}
    \only<1>{\item Chaque caractère tapé est directement inséré dans la copie locale;}
    \only<2>{\item
      \textbf{Chaque caractère tapé est directement inséré dans la copie locale;}}
    \only<3->{\item Chaque caractère tapé est directement inséré dans la copie locale;}
    \only<1-2>{\item La modification est disséminée à l'ensemble du réseau;
    \item Les éditeurs recevant la modification l'appliquent.}
    \only<3->{\item \textbf{La modification est disséminée à l'ensemble du réseau;}
    \item \textbf{Les éditeurs recevant la modification l'appliquent.}}
    \end{itemize}
  \end{minipage}
  \hfill
  \begin{minipage}{0.44\textwidth}
    \begin{center}
      
\begin{tikzpicture}[scale=1.1]

  \newcommand\X{30pt}
  \newcommand\Y{-30pt}
  
  \small

  \draw (-35-2*\X, 30pt);

  \draw[->] (\X, 0)--(\X, 5+3*\Y); %% p1 p6
  \draw[->] (-5+2*\X, 0)--(5+\X, 0); %% p2 p1
  \draw[->] (2*\X, 0) -- (-5+3*\X, \Y); %% p2 p3
  \draw[->] (2*\X, 0) -- (-5+3*\X, 2*\Y); %% p2 p4
  \draw[->] (3*\X, 5+\Y) -- ( 5+2*\X, 0); %% p3 p2
  \draw[->] (3*\X, \Y) -- (5pt, 2*\Y); %% p3 p7
  \draw[->] (3*\X, \Y) -- (2*\X, 5+3*\Y); %% p3 p5
  \draw[->] (3*\X, 2*\Y) -- (5pt, 2*\Y); %% p4 p7
  \draw[->] (3*\X, 2*\Y) -- (5pt, \Y); %% p4 p8
  \draw[->] (2*\X, 3*\Y) -- (\X, -5pt); %% p5 p1
  \draw[->] (5+2*\X, 3*\Y) -- (3*\X, -5+ 2*\Y); %% p5 p4
  \draw[->] (-5+\X, 3*\Y) -- (0pt, -5+2*\Y); %% p6 p7
  \draw[->] (0pt, 2*\Y) -- (-5+3*\X, \Y); %% p7 p3
  \draw[->] (0pt, 2*\Y) -- (\X, -5pt); %% p7 p1
  \draw[->] (0pt, \Y) -- (2*\X, -5pt); %% p8 p2
  \draw[->] (0pt, \Y) -- (\X, 5+3*\Y); %% p8 p6
  \draw[->] (0pt, \Y) -- (-5+3*\X, \Y); %% p8 p3
  
  \draw[fill=white] (-2*\X, 1.5*\Y) 
  node{\includegraphics[width=12pt]{img/crateicon.png}};
  % node{$e_9$}; % +(-5pt,-5pt) rectangle +(5pt,5pt);
  \only<3>{
    \draw[<->, densely dashed, color=darkblue, very thick]
    (-2*\X, 5+1.5*\Y) -- node[anchor=east]{\DARKBLUE{rejoint}} (-2*\X, -5pt);
  }

  \only<4>{
    \draw[<->, very thick](5-2*\X,1.5*\Y) -- (-5pt, -3+ \Y);
  }

  \draw[fill=white] (-2*\X, 0) node{$mediateur_1$} +(-20pt, -5pt)rectangle+(20pt, 5pt);
  \only<2->{
    \draw[<->, densely dashed, color=darkblue, very thick]
    (20-2*\X, 0) -- node[anchor=south]{\DARKBLUE{partage}} (-5+\X, 0);
    \draw[<->, densely dashed, color=darkblue, very thick]
    (20-2*\X, -5pt) -- (-5pt, \Y);
  }
  
  \draw[fill=white] (-2*\X, 3*\Y) node{$mediateur_2$} +(-20pt, -5pt)rectangle+(20pt, 5pt);
  \only<2->{
    \draw[<->, densely dashed, color=darkblue] (20-2*\X, 3*\Y) -- (-5+\X, 3*\Y);
    \draw[<->, densely dashed, color=darkblue] (20-2*\X, 5+3*\Y) -- (-5pt, 2*\Y);
  }

  \draw[fill=white] (\X, 0)node{\includegraphics[width=12pt]{img/crateicon.png}};
  % node{$e_1$}+(-5pt, -5pt)rectangle+(5pt, 5pt);
  \draw[fill=white] (2*\X, 0)node{\includegraphics[width=12pt]{img/crateicon.png}};
  % {$e_2$}+(-5pt, -5pt)rectangle+(5pt, 5pt);
  \draw[fill=white] (3*\X, \Y)node{\includegraphics[width=12pt]{img/crateicon.png}};
  % ;{$e_3$}+(-5pt, -5pt)rectangle+(5pt, 5pt);
  \draw[fill=white] (3*\X, 2*\Y)node{\includegraphics[width=12pt]{img/crateicon.png}};
  % {$e_4$}+(-5pt, -5pt)rectangle+(5pt, 5pt);
  \draw[fill=white] (2*\X, 3*\Y)node{\includegraphics[width=12pt]{img/crateicon.png}};
  % {$e_5$}+(-5pt, -5pt)rectangle+(5pt, 5pt);
  \draw[fill=white] (1*\X, 3*\Y)node{\includegraphics[width=12pt]{img/crateicon.png}};
  % {$e_6$}+(-5pt, -5pt)rectangle+(5pt, 5pt);
  \draw[fill=white] (0 , 2*\Y)node{\includegraphics[width=12pt]{img/crateicon.png}};
  % {$e_7$}+(-5pt, -5pt)rectangle+(5pt, 5pt);
  % \draw[fill=white] (0 , \Y)+(-55pt, -10pt)rectangle+(5pt, 10pt);
  \draw[fill=white](0,\Y) node{\includegraphics[width=12pt]{img/crateicon.png}};
  % {$e_8$}+(-5pt, -5pt)rectangle+(5pt, 5pt);

\end{tikzpicture}
    \end{center}
  \end{minipage}

  
  \vspace{1cm}
  \large
  \begin{itemize}
  \item [$\Rightarrow$] \textbf{taille de messages} $\times$ \textbf{nombre de messages}
  \end{itemize}

\end{frame}

\begin{frame}{Introduction}{Contributions}
    
  \begin{itemize}
  \item \textbf{taille des messages :} \LSEQ qui, dans le contexte de l'édition
    collaborative, borne la taille des messages de manière sous-linéaire par
    rapport au nombre d'insertions effectuées dans le document.
    \vspace{1cm}
  \item \textbf{nombre de messages :} \SPRAY qui s'adapte automatiquement à la
    taille du réseau de manière logarithmique et qui supporte le processus
    complexe d'établissement de connexion disponible dans les navigateurs Web.
  \end{itemize}

\end{frame}


%%% Local Variables:
%%% mode: latex
%%% TeX-master: "../slides"
%%% End:
