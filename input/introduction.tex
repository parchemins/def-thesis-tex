\section{Introduction}


\begin{frame}{Introduction}\framesubtitle{Édition collaborative}

  L'édition collaborative concerne toutes les activités effectuées en
  \textbf{groupe} dans le but de produire un \textbf{document}. L'effort
  collectif permet de bénéficier de multiples points de vues différents.

  \begin{itemize}
  \item[$\rightarrow$] Les documents sont de \textbf{meilleure qualité}.
  \end{itemize}
  
  \vspace{0.25cm}

  \noindent
  \begin{exampleblock}{Le Wikipédia anglais\ldots}
  \begin{minipage}{0.6\textwidth}
    \ldots compte \textbf{5 millions} d'articles, \textbf{40 millions}
      de pages et \textbf{112 mille} utilisateurs actifs.\\Les articles
      possèdent une \textbf{fiabilité} \textbf{comparable} à celle de
      l'Encyclopædia Britannica.
  \end{minipage}\footfullcite{giles2005internet}
  \hfill
  \begin{minipage}{0.3\textwidth}
    \begin{figure}
      \begin{center}
        \includegraphics[width=0.7\textwidth]{img/wikipedia.png}
      \end{center}
    \end{figure}
  \end{minipage}
  \end{exampleblock}

\end{frame}


\begin{frame}{Introduction}{Éditeur collaboratif Web}
%  \begin{minipage}{0.53\textwidth}
%     Un éditeur collaboratif permet
%     \begin{itemize}
%     \item à \textbf{plusieurs personnes}
%     \item de \textbf{lire} et \textbf{modifier} un document.
%       % \begin{itemize}
%       % \item \textbf{ajout} de caractères
%       % \item \textbf{suppression} de caractères
%       % \end{itemize}
%     \end{itemize}
    
%     Grâce au Web,
%     \begin{itemize}
%     \item n'importe quel outil accédant à l'internet (\textit{e.g. ordinateur,
%         smartphone, tablette}) permet de créer et d'éditer un document aisément.
%     \item Un simple lien permet de le partager facilement avec des amis ou des
%       collègues.
%     \end{itemize}    
% %  \end{minipage}
% %  \begin{minipage}{0.45\textwidth}
%     \begin{figure}    
%       \begin{center}
%         \includegraphics[width=0.65\textwidth]{img/googledocs.png}
% %        \caption{Capture d'écran d'un document Google Docs rédigé par 3 personnes
% %          en simultané. \REF}
%       \end{center}
%     \end{figure}
%  \end{minipage}


%  \begin{figure}   
%    \hspace{-1cm}
    \begin{textblock*}{\textwidth}(-0.8cm,-2cm) 
    \includegraphics[width=1.13\textwidth]{img/googledocs3.png}\footfullcite{johansen1988groupware}
  \end{textblock*}
 % \end{figure}
\end{frame}


\begin{frame}{Introduction}{Problématique}

  % L'organisation du Web est éminemment centralisée : quelques serveurs sont en
  % charge d'un nombre titanesque de clients.
  % \begin{itemize}
  % \item Problèmes de confidentialité, de censure, d'intelligence économique, de
  %   propriété, etc.
  % \item Problèmes de passage à l'échelle et de résilience aux pannes.
  % \end{itemize}
  
  \begin{minipage}{0.69\textwidth}
    Le contexte \textbf{Web} pousse à la \textbf{centralisation :}
    \begin{itemize}
    \item problèmes de \textbf{confidentialité}, \textbf{censure}, etc.
    \uncover<2->{\item problèmes de passage à l'échelle, notamment en \textbf{nombre de
        collaborateurs};}
    \uncover<3->{\item problèmes de \textbf{résilience} aux pannes.}
    \end{itemize}
  \end{minipage}
  \hfill
  \begin{minipage}{0.3\textwidth}
    \includegraphics[width=0.7\textwidth]{img/www.png}
  \end{minipage}

%  \vspace{0.025cm}
  

  \begin{minipage}{0.32\textwidth}
    \begin{center}
      \begin{tikzpicture}
        \node[visible on=<1-3>]
        {\includegraphics[width=0.95\textwidth]{img/centralizedethicproblems.png}};
      \end{tikzpicture}
    \end{center}
  \end{minipage}
  \begin{minipage}{0.32\textwidth}
    \begin{center}
      \begin{tikzpicture}
        \node[visible on=<2-3>]
        {\includegraphics[width=0.95\textwidth]{img/centralizedcpuproblems.png}};
      \end{tikzpicture}
    \end{center}
  \end{minipage}
  \begin{minipage}{0.32\textwidth}
    \begin{center}
      \begin{tikzpicture}
        \node[visible on=<3-3>]
        {\includegraphics[width=0.95\textwidth]{img/centralizedscalabilityproblems.png}};
      \end{tikzpicture}        
    \end{center}
  \end{minipage}

%  \vspace{0.025cm}

  \uncover<2->{\begin{exampleblock}{En 2013, Coursera rassembla 41000 étudiants\ldots}
    \ldots pour un seul cours.  Les cours étant basés sur les outils de
    collaboration Google, ils n'offraient la collaboration en temps réel qu'à un
    petit groupe d'utilisateurs. Selon les journalistes, l'expérience fut un \og
    désastre\fg \footfullcite{strauss2013how}.
  \end{exampleblock}}

  \vspace{0.05cm}
\end{frame}


\begin{frame}{Introduction}\framesubtitle{Éditeur collaboratif décentralisé}
  
  \begin{itemize}
  \item [$\Rightarrow$] \large\textbf{L'édition collaborative temps réel est-elle
      possible sur le Web, sans l'intervention d'un tiers et sans limites quant
      aux dimensions du système ?}
    \begin{itemize}
    \item [$\rightarrow$] Décentralisé large échelle dans les navigateurs Web.
    \end{itemize}
  \end{itemize}
  
  \vspace{0.25cm}
  \begin{center}
    \includegraphics[width=0.8\textwidth]{img/world.png}
  \end{center}
  \vspace{0.25cm}
  
  \begin{enumerate}
    \item Moyen de représenter efficacement un \textbf{document} de manière
      \textbf{cohérente};
    \item Moyen de \textbf{communiquer} efficacement les modifications sur le
      document.
  \end{enumerate}

  % \begin{itemize}
  % \item Répartition géographique des collaborateurs;
  % \item Édition en temps réel.
  % \end{itemize}
\end{frame}


%%% Local Variables:
%%% mode: latex
%%% TeX-master: "../slides"
%%% End:
